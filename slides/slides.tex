% ----------------------------------------------------------------------------------------------------------------------------------
% Efficiently Backing up Terabytes of Data with pgBackRest
%
% Build from the Vagrant VM:
% cd /talk/slides && make -f /template/Makefile
% ----------------------------------------------------------------------------------------------------------------------------------
\def\mytitle{Efficiently Backing up Terabytes of Data with pgBackRest}
\def\mysubject{}
\def\myevent{PostgresOpen 2016}
\def\myauthor{David Steele}
\def\myemail{}
\def\mydate{September 14, 2016}

% Suppres navigation bars
\def\mysuppressnav{}

% Include Crunchy template
\def\mytemplatepath{/template/}
\input{\mytemplatepath crunchy-template.tex}

% Agenda
\begin{frame}
    \frametitle{Agenda}
    \tableofcontents
\end{frame}

\section{Why Backup?}

\begin{frame}
    \frametitle{Why Backup?}

    \begin{itemize}
        \item Hardware Failure:

        \begin{itemize}
            \item No amount of redundancy can prevent it.\pause
        \end{itemize}

        \item Replication:

        \begin{itemize}
            \item WAL archive for when async streaming gets behind.\pause
            \item Sync replica from backup instead of master.\pause
        \end{itemize}

        \item Corruption:

        \begin{itemize}
            \item Can be caused by hardware or software.\pause
            \item Detection is of course a challenge.
        \end{itemize}
    \end{itemize}
\end{frame}

\begin{frame}
    \frametitle{Why Backup?}

    \begin{itemize}
        \item Accidents:

        \begin{itemize}
            \item So you dropped a table?\pause
            \item Deleted your most important account?\pause
        \end{itemize}

        \item Development:

        \begin{itemize}
            \item No more realistic data than production!\pause
            \item May not be practical due to size / privacy issues.\pause
        \end{itemize}

        \item Reporting:

        \begin{itemize}
            \item Use backups to standup an independent reporting server.\pause
            \item Recover important data that was removed on purpose.
        \end{itemize}
    \end{itemize}
\end{frame}

\section{Living Backups}

\begin{frame}
    \frametitle{Schr\"{o}dinger’s Backup}

    The state of any backup is unknown until a restore is attempted.
\end{frame}

\begin{frame}
    \frametitle{Making Backups Useful}

    \begin{itemize}
        \item Find a way to use your backups

        \begin{itemize}
            \item Syncing / New Replicas
            \item Offline reporting
            \item Offline data archiving
            \item Development
        \end{itemize}

        \item Unused code paths will not work when you need them unless they are tested

        \begin{itemize}
            \item Regularly scheduled automated failover using backups to restore the old primary
            \item Regularly scheduled disaster recovery (during a maintenance window if possible) to test restore techniques
        \end{itemize}
    \end{itemize}
\end{frame}

% \section{How to Backup?}
%
% \begin{frame}
%     \frametitle{How to Backup?}
%
%     \begin{itemize}
%         \item pg\_dump
%
%         \item pg\_basebackup
%
%         \item Manual
%
%         \item Third Party
%
%         \begin{itemize}
%             \item OmniPITR
%             \item Barman
%             \item WAL-E
%         \end{itemize}
%
%         \item pgBackRest!
%     \end{itemize}
% \end{frame}

\section{Design}

\begin{frame}
    \frametitle{Design}

    \begin{itemize}
        \item Rsync powers many database backup solutions but it has some serious limitations:\pause

        \begin{itemize}
            \item Single-process.\pause
            \item One second timestamp resolution.\pause
            \item Incremental backups require previous backup to be uncompressed.\pause
        \end{itemize}

        \item pgBackRest does not use rsync, tar or other typical backup tools:\pause

        \begin{itemize}
            \item Protocol supports local/remote operation.\pause
            \item Solves timestamp resolution issue.
        \end{itemize}
    \end{itemize}
\end{frame}

\section{Features}

\begin{frame}
    \frametitle{Multi-Process Backup \& Restore}

    \begin{itemize}
        \item Compression is usual bottleneck:\pause

        \begin{itemize}
            \item But most PostgreSQL backup solutions are single-process.\pause
            \item pgBackRest solves the problem with multi-processing.\pause
            \item 1TB/hr raw throughput even on a 1Gb/s link using multiple cores.
        \end{itemize}
    \end{itemize}
\end{frame}

\begin{frame}
    \frametitle{Local or Remote Operation}

    \begin{itemize}
        \item Custom protocol allows backup, restore, and archive locally or remotely via SSH with minimal configuration.\pause
        \item No direct access to PostgreSQL is required from the remote server which enhances security.
    \end{itemize}
\end{frame}

\begin{frame}
    \frametitle{Full, Incremental, \& Differential Backups}

    \begin{itemize}
        \item Multiple backup types:\pause

            \begin{itemize}
                \item Full\pause
                \item Differential\pause
                \item Incremental\pause
            \end{itemize}

        \item pgBackRest is not susceptible to the time resolution issues of rsync, making differential and incremental backups safe.
    \end{itemize}
\end{frame}

\begin{frame}
    \frametitle{Backup Rotation \& Archive Expiration}

    \begin{itemize}
        \item Retention Based on full or differential backups.\pause
        \item WAL retention for all backups or configure number of recent backups.\pause
        \item WAL required for consistency of backups always preserved.
    \end{itemize}
\end{frame}

\begin{frame}
    \frametitle{Backup Integrity}

    \begin{itemize}
        \item Checksums are calculated for every file in the backup and rechecked during a restore.\pause
        \item After a backup required WAL segments are checked in the repository.\pause
        \item Simple backup format:\pause

            \begin{itemize}
                \item Backup directories have the same format as a PostgreSQL cluster.\pause
                \item Clusters can be brought up in place with snapshots if compression is disabled.\pause
                \item Advantageous for terabyte-scale databases.\pause
            \end{itemize}

        \item All operations utilize file and directory level fsync to ensure durability.
    \end{itemize}
\end{frame}

\begin{frame}
    \frametitle{Backup Resume}

    \begin{itemize}
        \item An aborted backup can be resumed from the point where it stopped.\pause
        \item Checksumming files on resume takes place on the backup server.\pause
        \item Saves load on the master by not compressing and transmitting resumed files.
    \end{itemize}
\end{frame}

\begin{frame}
    \frametitle{Streaming Compression \& Checksums}

    \begin{itemize}
        \item Compression and checksum calculations are performed in stream.\pause
        \item Compression is not done more than once.\pause
        \item Lower compression is used when the destination is uncompressed to efficiently utilize CPU and network bandwidth.
    \end{itemize}
\end{frame}

\begin{frame}
    \frametitle{Delta Restore}

    \begin{itemize}
        \item Backup manifest contains checksum and size for every file.\pause
        \item On delta restore all files not present in the backup or with a different size are removed from PGDATA.\pause
        \item The remaining files are checksummed and only files with a checksum mismatch are restored.\pause
        \item Multi-processing can lead to dramatic reductions in restore time and network utilization.
    \end{itemize}
\end{frame}

\begin{frame}
    \frametitle{Advanced Archiving}

    \begin{itemize}
        \item Dedicated commands are included for both pushing WAL to the archive and retrieving WAL from the archive.\pause
        \item Push command automatically detects WAL segments that are pushed multiple times and de-duplicates when the segment is identical, otherwise an error is raised.\pause
        \item Push and get commands both ensure that the database and repository match by comparing PostgreSQL versions and system identifiers to prevent misconfiguration.\pause

        \item Asynchronous archiving allows compression and transfer to be offloaded to another process which maintains a continuous connection to the remote server, improving throughput significantly.\pause

            \begin{itemize}
                \item Critical feature for databases with extremely high write volume.
            \end{itemize}
    \end{itemize}
\end{frame}

\begin{frame}
    \frametitle{Tablespace \& Link Support}

    \begin{itemize}
        \item Tablespaces are fully supported and on restore tablespaces can be remapped to any location.\pause
        \item Remap all tablespaces to one location with a single command which is useful for development restores.\pause
    \end{itemize}

    \begin{itemize}
        \item File and directory links are supported for any file or directory in the PostgreSQL cluster.\pause
        \item Restore all links to their original locations, remap some or all links, or restore some or all links as normal files or directories within the cluster directory
    \end{itemize}
\end{frame}

\begin{frame}
    \frametitle{Selective Restore}

    \begin{itemize}
        \item Restore only specified databases out of a cluster backup.\pause
        \item Other files are restored as sparse, zeroed files the save space.\pause
        \item All WAL must be replayed.\pause
        \item Cannot connect to non-restored databases, can only drop them.\pause
    \end{itemize}
\end{frame}

\begin{frame}
    \frametitle{Backup from Standby}

    \begin{itemize}
        \item Backup is started on master.\pause
        \item Backup starts when replay location on standby reaches start backup location.\pause
        \item Reduces load on master because replicated files are copied from the standby.
    \end{itemize}
\end{frame}

\begin{frame}
    \frametitle{Compatibility with PostgreSQL $\geqslant$ 8.3}

    \begin{itemize}
        \item Support for versions down to 8.3, since older versions of PostgreSQL are still regularly utilized.
    \end{itemize}
\end{frame}

\section{Performance}

\begin{frame}
    \frametitle{Performance}

    \begin{table}
        \begin{tabular}{l || c | c}
            \bf{Parameters} & \bf{pgBackRest} & \bf{rsync} \\
            \hline \hline
            \pbox{14em}{processes: 1 \\ network compression: l3 \\ destination compression: none} &
                141 Seconds &
                \pbox{14em}{124 Seconds \\ (.13X Faster)} \\ \hline
            \pbox{14em}{processes: 2 \\ network compression: l3 \\ destination compression: none} &
                \pbox{14em}{84 Seconds \\ (1.48X Faster)} &
                N/A \\ \hline
            \pbox{14em}{processes: 1 \\ network compression: l6 \\ destination compression: l6} &
                \pbox{14em}{334 Seconds \\ (1.52X Faster)} &
                510 Seconds \\ \hline
            \pbox{14em}{processes: 2 \\ network compression: l6 \\ destination compression: l6} &
                \pbox{14em}{174 Seconds \\ (2.93X Faster)} &
                N/A
        \end{tabular}
    \end{table}

\end{frame}

\section{Demonstration}

\begin{frame}
    \frametitle{Demonstration}

    Live Demo --- this should be fun!
\end{frame}

\section{Questions?}

\begin{frame}
    \frametitle{Questions?}

    website: \url{http://www.pgbackrest.org}\\
    \vspace{1em}
    email: \href{mailto:david@pgbackrest.org}{david@pgbackrest.org} \\
    email: \href{mailto:david@crunchydata.com}{david@crunchydata.com}\\
    \vspace{1em}
    releases: \url{https://github.com/pgbackrest/pgbackrest/releases}\\
    \vspace{1em}
    slides \& demo: \url{https://github.com/dwsteele/conference/releases}\\
\end{frame}

% End document
\end{document}
