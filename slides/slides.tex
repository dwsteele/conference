% ----------------------------------------------------------------------------------------------------------------------------------
% Backup Best Practices with pgBackRest
%
% Build from the Vagrant VM:
% cd /talk/slides && make -f /template/Makefile
% ----------------------------------------------------------------------------------------------------------------------------------
\def\mytitle{Backup Best Practices\\with pgBackRest}
\def\mysubject{}
\def\myevent{PostgresConf.CN \& PGConf.Asia}
\def\myauthor{David Steele}
\def\myemail{}
\def\mydate{November 20, 2020}

% Suppres navigation bars
\def\mysuppressnav{}

% Include Crunchy template
\def\mytemplatepath{/template/}
\input{\mytemplatepath crunchy-template.tex}

% Agenda
\begin{frame}
    \frametitle{Agenda}
    \tableofcontents
\end{frame}

% ----------------------------------------------------------------------------------------------------------------------------------
\section{Introduction}

\begin{frame}
    \frametitle{About the Speaker}

    \begin{itemize}
        \item Principal Architect at Crunchy Data, the Trusted Open Source Enterprise PostgreSQL Leader.
        \item Actively developing with PostgreSQL since 1999.
        \item Maintainer of pgBackRest and pgAudit.
        \item PostgreSQL Contributor.
    \end{itemize}
\end{frame}

% ----------------------------------------------------------------------------------------------------------------------------------
\section{Why and How to Backup}

\begin{frame}
    \frametitle{Why Backup?}

    \begin{itemize}
        \item Hardware Failure:

        \begin{itemize}
            \item No amount of redundancy can prevent it.
        \end{itemize}

        \item Replication:

        \begin{itemize}
            \item WAL archive for when async streaming gets behind.
            \item Sync replica from a backup instead of the primary.
        \end{itemize}

        \item Corruption:

        \begin{itemize}
            \item Can be caused by hardware or software.
            \item Detection is, of course, a challenge.
        \end{itemize}
    \end{itemize}
\end{frame}

\begin{frame}
    \frametitle{Why Backup?}

    \begin{itemize}
        \item Accidents:

        \begin{itemize}
            \item So you dropped a table?
            \item Deleted your most important account?
        \end{itemize}

        \item Development:

        \begin{itemize}
            \item No more realistic data than production!
            \item May not be practical due to size / privacy issues.
        \end{itemize}

        \item Reporting:

        \begin{itemize}
            \item Use backups to standup an independent reporting server.
            \item Recover important data that was removed on purpose.
        \end{itemize}
    \end{itemize}
\end{frame}

\begin{frame}
    \frametitle{Making Backups Useful}

    \begin{itemize}
        \item Find a way to use your backups

        \begin{itemize}
            \item Syncing / New Replicas
            \item Offline reporting
            \item Offline data archiving
            \item Development
        \end{itemize}

        \item Unused code paths will not work when you need them unless they are tested

        \begin{itemize}
            \item Regularly scheduled automated failover using backups to restore the old primary
            \item Regularly scheduled disaster recovery (during a maintenance window if possible) to test restore techniques
        \end{itemize}
    \end{itemize}
\end{frame}

% ----------------------------------------------------------------------------------------------------------------------------------
\section{What is pgBackRest?}

\begin{frame}
    \frametitle{What is pgBackRest?}

    pgBackRest aims to be a reliable, easy-to-use backup and restore solution that can seamlessly scale up to the largest databases and workloads by utilizing algorithms that are optimized for database-specific requirements.
    \\\vspace{1em}
    Major features:

    \begin{itemize}
        \item Parallelism and asynchronous operation
        \item Full, incremental, and differential backups
        \item Backup and archive expiration
        \item Integrity checks / page checksum verification
        \item Delta restore
        \item S3/Azure compatible object store support
        \item Encryption
        \item Multiple compression types (gzip, bzip, lz4, zstd)
    \end{itemize}

    The pgBackRest project started in 2013 and was originally written in Perl. The migration to pure C was completed in 2019.
\end{frame}

% ----------------------------------------------------------------------------------------------------------------------------------
\section{Backup Types and Retention}

\begin{frame}
    \frametitle{Backup Types}

    \begin{itemize}
        \item Full

        A complete copy of the database, not dependent on any other backup.

        \item Incremental

        Copy only files that have changed since the last backup (full, diff, or incr).

        \item Differential

        Like an incremental, but only copy files that have changed since the last \textbf{full} backup.
    \end{itemize}

    Incremental is the default since it is generally the least expensive backup type in terms of time and space.
\end{frame}

\begin{frame}[fragile]
    \frametitle{Backup Retention (Count-based)}

    A good default retention is:

    \begin{quote}\begin{verbatim}
repo1-retention-full=5
repo1-retention-diff=3
    \end{verbatim}\end{quote}\vspace{-1em}

    Run the full backup once a week and differential backups the other six days:

    \begin{quote}\begin{verbatim}
# m h  dom mon dow  command
45 08  *   *   0    pgbackrest --type=full --stanza=demo backup
45 08  *   *   1-6  pgbackrest --type=diff --stanza=demo backup
    \end{verbatim}\end{quote}\vspace{-1em}

    With this retention there will always be at least five full backups.
    \\\vspace{1em}
    If full backups are run weekly then there will be four weeks of retention but if they are run more frequently then retention will be reduced.
\end{frame}

\begin{frame}[fragile]
    \frametitle{Backup Retention (Time-based)}

    A safer option is to use time-based retention.
    \\\vspace{1em}
    Four weeks of full retention with three differentials:

    \begin{quote}\begin{verbatim}
repo1-retention-full-type=time
repo1-retention-full=28
repo1-retention-diff=3
    \end{verbatim}\end{quote}\vspace{-1em}

    Run the full backup once a week and differentials the other six days.
    \\\vspace{1em}
    With this schedule there will always be at least 4 weeks of backups and accidentally making an extra full backup will not reduce the retention period.
\end{frame}

\begin{frame}[fragile]
    \frametitle{Backup Retention (Ad hoc)}

    If there are extra, unneeded backups they can be pruned with ad hoc expiration:

    \begin{quote}\begin{verbatim}
pgbackrest --stanza=demo --set=20201103_190545F expire
    \end{verbatim}\end{quote}\vspace{-1em}

    Note that dependent backups will also be expired, i.e. differential and incremental backups that depend on the full backup.
    \\\vspace{1em}
    Use the \texttt{--dry-run} option to see which backups will be expired.
    \\\vspace{1em}
    Combining the \texttt{info} command and ad hoc expiration it is possible to create a custom retention schedule, though this would require some coding.
\end{frame}

% ----------------------------------------------------------------------------------------------------------------------------------
\section{Archiving}

\begin{frame}
    \frametitle{Archiving}

    Asynchronous parallel archiving allows compression and transfer to be offloaded to another process which maintains continuous connections to the remote server, improving throughput significantly.
    \\\vspace{1em}
    Make sure there is enough space to hold WAL for an extended period if archiving stops.
    \\\vspace{1em}
    It is very important to monitor archiving to ensure it continues working.
\end{frame}

% ----------------------------------------------------------------------------------------------------------------------------------
\section{Recovery Point Objective}

\begin{frame}[fragile]
    \frametitle{Recovery Point Objective}

    \Large A recovery point objective (RPO) is the maximum acceptable amount of data loss measured in time.
\end{frame}

\begin{frame}[fragile]
    \frametitle{Meeting RPO}

    First, define the RPO. If the RPO is zero then synchronous replication may be the only viable option. Note that synchronous replication can have major performance impacts, especially in high-latency environments, e.g. multiple data centers.
    \\\vspace{1em}
    In certain situations, a zero RPO may not be possible, e.g., a table is dropped from the primary.
    \\\vspace{1em}
    For a non-zero RPO there are a few good options:

    \begin{itemize}
        \item Asynchronous replication

        Replication lag should be monitored to ensure it does not fall outside the RPO requirement.

        \item Archive timeout

        The \texttt{archive\_timeout} setting ensures that a WAL segment is archived at the interval specified which can be tuned to meet RPO.
    \end{itemize}

    These two methods can be combined to minimize RPO within the allowance of the RTO.
\end{frame}

% ----------------------------------------------------------------------------------------------------------------------------------
\section{Backup}

\begin{frame}
    \frametitle{Backup from Standby}

    Backup from standby greatly reduces load on the primary.
    \\\vspace{1em}
    pgBackRest uses a hybrid technique that uses both the primary and a standby to make a backup:

    \begin{itemize}
        \item Backup is started on primary.
        \item Backup starts when replay location on standby reaches start backup location.
        \item Reduces load on the primary because replicated files are copied from the standby.
    \end{itemize}
\end{frame}

\begin{frame}
    \frametitle{Backup to a Safe Location}

    Do not backup to storage on the PostgreSQL host!
    \\\vspace{1em}
    Instead, backup to:
    \begin{itemize}
        \item Storage on a repository host.
        \item NFS (or other network storage) mounted to the PostgreSQL host.
        \item An object store (S3, Azure).
    \end{itemize}
\end{frame}

\begin{frame}
    \frametitle{Page Checksums}

    When initializing a new cluster always specify the \texttt{-k} option.
    \\\vspace{1em}
    This will enable page checksums which pgBackRest will verify on every backup. If corruption is detected early it can often be corrected.
    \\\vspace{1em}
    If corruption is detected, don't panic! As long as older non-corrupted backups exist there are options:

    \begin{itemize}
        \item Restore an older backup and perform recovery

        There is a good chance that corruption in the database is not in the WAL.

        \item Restore an older backup and retrieve corrupted data

        If corruption exists in only one area then it may be possible to retrieve that data from a backup and then load it into the primary database.
    \end{itemize}
\end{frame}

\begin{frame}
    \frametitle{Failed Backups}

    A failed backup should not be cause for panic. Ideally, there are a number of backups that can be used for a restore.
    \\\vspace{1em}
    Remember:
    \\\vspace{1em}
    Backup + WAL = Recoverable Database
    \\\vspace{1em}
    and:
    \\\vspace{1em}
    \textbf{OLDER} Backup + WAL = Recoverable Database
\end{frame}

% ----------------------------------------------------------------------------------------------------------------------------------
\section{Recovery Time Objective}

\begin{frame}[fragile]
    \frametitle{Recovery Time Objective}

    \Large The Recovery Time Objective (RTO) is the maximum time allowed between an unexpected failure and the resumption of normal operation.
\end{frame}

\begin{frame}[fragile]
    \frametitle{Meeting the Recovery Time Objective}

    First, define the Recovery Time Objective (RTO).
    \\\vspace{1em}
    The easiest way to meet a low RTO is to have a standby server ready and failover from the primary. This can be done manually or with a tool that automates the process, e.g. Patroni.
    \\\vspace{1em}
    Monitor the standby to make sure replication lag is within the tolerance of your Recovery Point Objective (RPO).
    \\\vspace{1em}
    A standby is not a backup! Even with standby it is important to have recent backups in case there are problems with the standby, or there was data loss that has already been replicated.
    \\\vspace{1em}
    Daily backups are usually sufficient, but be sure to measure your restore and recovery time to be sure it meets RTO.
\end{frame}

% ----------------------------------------------------------------------------------------------------------------------------------
\section{Restore and Recovery}

\begin{frame}
    \frametitle{Schr\"{o}dinger’s Backup}

    \Large The state of any backup is unknown until a restore is attempted.
\end{frame}

\begin{frame}
    \frametitle{Restore and Recovery}

    Multi-processing can lead to dramatic reductions in restore time and network utilization.
    \\\vspace{1em}
    Use the \texttt{--delta} option with the \texttt{restore} command to save time.
    \\\vspace{1em}
    Double-check where you are restoring. pgBackRest will refuse to overwrite a \textit{running} cluster but it will overwrite a shut down cluster when \texttt{--delta} is used.
\end{frame}

\begin{frame}
    \frametitle{Restore Testing}

    Test your restores! Also validate that the database looks correct after the restore.

    \begin{itemize}
        \item Check timestamps in a database to verify RPO was met
        \item Use tools such as amcheck to verify consistency of the cluster
        \item Run \texttt{pg\_dump} and (if possible) restore the cluster from the dump
        \item Run smoke tests and/or regression tests
    \end{itemize}
\end{frame}

\begin{frame}
    \frametitle{Make a Disaster Recovery Plan}

    \Large Make a disaster recovery plan.
    \\\vspace{1em}
    \Large Document the disaster recovery plan.
    \\\vspace{1em}
    \Large Practice the disaster recovery plan.
\end{frame}

\section{Questions?}

\begin{frame}
    \frametitle{Questions?}

    website: \url{http://www.pgbackrest.org}\\
    \vspace{1em}
    email: \href{mailto:david@pgbackrest.org}{david@pgbackrest.org} \\
    email: \href{mailto:david@crunchydata.com}{david@crunchydata.com}\\
    \vspace{1em}
    releases: \url{https://github.com/pgbackrest/pgbackrest/releases}\\
    \vspace{1em}
    slides \& demo: \url{https://github.com/dwsteele/conference/releases}\\
\end{frame}

% End document
\end{document}
