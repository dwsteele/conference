% ----------------------------------------------------------------------------------------------------------------------------------
% High Performance pgBackRest
%
% Build from the Vagrant VM:
% cd /talk/slides && make -f /template/Makefile
% ----------------------------------------------------------------------------------------------------------------------------------
\def\mytitle{High Performance pgBackRest}
\def\mysubject{}
\def\myevent{Montr\'eal PostgreSQL Meetup}
\def\myauthor{David Steele}
\def\myemail{}
\def\mydate{July 16, 2019}

% Suppres navigation bars
\def\mysuppressnav{}

% Include Crunchy template
\def\mytemplatepath{/template/}
\input{\mytemplatepath crunchy-template.tex}

% Agenda
\begin{frame}
    \frametitle{Agenda}
    \tableofcontents
\end{frame}

\section{Introduction}

\begin{frame}
    \frametitle{About the Speaker}

    \begin{itemize}
        \item Principal Architect at Crunchy Data, the Trusted Open Source Enterprise PostgreSQL Leader.
        \item Actively developing with PostgreSQL since 1999.
        \item Primary author of pgBackRest and co-author of pgAudit.
        \item PostgreSQL Contributor.
    \end{itemize}
\end{frame}

\begin{frame}
    \frametitle{What is pgBackRest?}

    pgBackRest aims to be a simple, reliable backup and restore system that can seamlessly scale up to the largest databases and workloads.\pause\vspace{1em}

    pgBackRest has a strong emphasis on performance, including:

    \begin{itemize}
        \item Parallel/asynchronous operation for all core commands\pause
        \item Backup from Standby\pause
        \item Advanced configuration for tuning specific commands
    \end{itemize}
\end{frame}

\section{Core Commands}

\begin{frame}
    \frametitle{Core Commands}

    \begin{itemize}
        \item Archive Push \\\vspace{1em}

        Allows PostgreSQL to push a completed WAL segment to the repository.\pause\vspace{1em}

        \item Backup \\\vspace{1em}

        Backup a PostgreSQL cluster.\pause\vspace{1em}

        \item Archive Get \\\vspace{1em}

        Allows PostgreSQL to get a completed WAL segment from the repository.\pause\vspace{1em}

        \item Restore \\\vspace{1em}

        Restore a PostgreSQL cluster.
    \end{itemize}
\end{frame}

\section{Archive Push}

\begin{frame}
    \frametitle{Archive Push Features}

    \begin{itemize}
        \item Asynchronous operation

        \begin{itemize}
            \item Asynchronously scan the \texttt{archive\_status} directory for WAL segments that are ready to be archived.\pause
            \item Store status of each WAL segment locally so PostgreSQL can be notified via the \texttt{archive\_command} of success or failure.\pause
            \item Written in pure C for performance.\pause
        \end{itemize}

        \item Parallelism

        \begin{itemize}
            \item Checksum, compress, encrypt, and transfer in parallel to improve throughput.
        \end{itemize}
    \end{itemize}
\end{frame}

\begin{frame}[fragile]
    \frametitle{Archive Push Configuration}

    \vspace{.75em}\begin{lstlisting}[title=pgbackrest.conf]
[global:archive-push]
archive-async=y
process-max=4
spool-path=/path/to/spool
    \end{lstlisting}\pause\vspace{1em}

    \begin{itemize}
        \item The \texttt{spool-path} parameter is optional (defaults to \texttt{/var/spool/pgbackrest}).\pause
        \item The spool directory must exist for asynchronous operation.\pause
        \item Note that configuration may be done with environment variables, e.g. \texttt{PGBACKREST\_ARCHIVE\_ASYNC}, or the command-line, e.g. \texttt{--archive-async}.
    \end{itemize}
\end{frame}

\section{Backup}

\begin{frame}
    \frametitle{Backup Features}

    \begin{itemize}
        \item Backup from Standby

        \begin{itemize}
            \item Perform most of the backup from a standby to reduce load on the primary.\pause
            \item Primary and standby are automatically selected from a list of clusters.\pause
        \end{itemize}

        \item Parallelism

        \begin{itemize}
            \item Checksum, compress, encrypt, and transfer in parallel to improve throughput.
        \end{itemize}
    \end{itemize}
\end{frame}

\begin{frame}[fragile]
    \frametitle{Backup Configuration}

    \vspace{.75em}\begin{lstlisting}[title=pgbackrest.conf]
[global:backup]
backup-standby=y
process-max=8

[demo]
pg1-host=pg1
pg1-path=/var/lib/postgresql/10
pg2-host=pg2
pg2-path=/var/lib/postgresql/10
pg3-host=pg3
pg3-path=/var/lib/postgresql/10
    \end{lstlisting}\pause\vspace{1em}

    \begin{itemize}
        \item The current primary can be in any position in the list of PostgreSQL servers.\pause
        \item The first live standby found will be used to perform the backup.
    \end{itemize}
\end{frame}

\section{Archive Get}

\begin{frame}
    \frametitle{Archive Get Features}

    \begin{itemize}
        \item Asynchronous operation

        \begin{itemize}
            \item Asynchronously build a queue of WAL segments that PostgreSQL will need.\pause
            \item Move or copy segments from the queue when requested by \texttt{restore\_command}.\pause
            \item The spool directory should be located on the same device as \texttt{pg\_xlog}/\texttt{pg\_wal} for best performance.\pause
            \item Written in pure C for performance.\pause
        \end{itemize}

        \item Parallelism

        \begin{itemize}
            \item Transfer, decrypt, decompress, and checksum in parallel to improve throughput.
        \end{itemize}
    \end{itemize}
\end{frame}

\begin{frame}[fragile]
    \frametitle{Archive Get Configuration}

    \vspace{.75em}\begin{lstlisting}[title=pgbackrest.conf]
[global:archive-get]
archive-async=y
archive-get-queue-max=1GB
process-max=2
    \end{lstlisting}\pause\vspace{1em}

    \begin{itemize}
        \item Archive Get generally requires fewer processes than Archive Push because decompression is less CPU-intensive than compression.\pause
        \item On the other hand, clusters in recovery have more CPU resources to spare.\pause
        \item The idea is to keep PostgreSQL supplied with WAL so that it doesn't need to wait.
    \end{itemize}
\end{frame}

\section{Restore}

\begin{frame}
    \frametitle{Restore Features}

    Restore performance is far more important than backup performance!\pause

    \begin{itemize}
        \item Delta operation

        \begin{itemize}
            \item Checksum local cluster files to determine what can be preserved.\pause
            \item Transfer only files that have changed since the last backup from the repository.\pause
        \end{itemize}

        \item Parallelism

        \begin{itemize}
            \item Transfer, decrypt, decompress, and checksum in parallel to improve throughput.
        \end{itemize}
    \end{itemize}
\end{frame}

\begin{frame}[fragile]
    \frametitle{Restore Configuration}

    \vspace{.75em}\begin{lstlisting}[title=pgbackrest.conf]
[global:restore]
process-max=16
delta=y
    \end{lstlisting}\vspace{1em}
\end{frame}

\section{Other Considerations}

\begin{frame}
    \frametitle{High Latency}

    The \texttt{process-max} option can be used to speed transfers on high latency storage such as S3.
\end{frame}

\begin{frame}
    \frametitle{Compression}

    The \texttt{compress-level} option can be lowered (e.g. \texttt{6} to \texttt{3}) to reduce the CPU cost of compression. This also reduces the compression ratio, but the time savings are often worth it.\pause

    \vspace{1em}We are introducing \texttt{lz4} support soon for a faster alternative to \texttt{gzip}.
\end{frame}

\begin{frame}
    \frametitle{The Future}

    The entire project will be migrated to C by the end of 2019.  We are adding in many performance enhancements as we go.
\end{frame}

\section{Questions?}

\begin{frame}
    \frametitle{Questions?}

    website: \url{http://www.pgbackrest.org}\\
    \vspace{1em}
    email: \href{mailto:david@pgbackrest.org}{david@pgbackrest.org} \\
    email: \href{mailto:david@crunchydata.com}{david@crunchydata.com}\\
    \vspace{1em}
    releases: \url{https://github.com/pgbackrest/pgbackrest/releases}\\
    \vspace{1em}
    slides: \url{https://github.com/dwsteele/conference/releases}\\
\end{frame}

% End document
\end{document}
