% ----------------------------------------------------------------------------------------------------------------------------------
% Evolving towards Postgres
%
% Build from the Vagrant VM:
% cd /talk/slides && make -f /template/Makefile
% ----------------------------------------------------------------------------------------------------------------------------------
\def\mytitle{Evolving towards Postgres}
\def\mysubject{}
\def\myevent{PostgresOpen}
\def\myauthor{David Steele}
\def\myemail{}
\def\mydate{September 13, 2019}

% Suppres navigation bars
\def\mysuppressnav{}

% Include Crunchy template
\def\mytemplatepath{/template/}
\input{\mytemplatepath crunchy-template.tex}

% Agenda
\begin{frame}
    \frametitle{Agenda}
    \tableofcontents
\end{frame}

\section{Introduction}

\begin{frame}
    \frametitle{About the Speaker}

    \begin{itemize}
        \item Principal Architect at Crunchy Data, the Trusted Open Source Enterprise PostgreSQL Leader.
        \item Actively developing with PostgreSQL since 1999.
        \item Primary author of pgBackRest and co-author of pgAudit.
        \item PostgreSQL Contributor.
    \end{itemize}
\end{frame}

\begin{frame}
    \frametitle{What is pgBackRest?}

    pgBackRest aims to be a simple, reliable backup and restore system that can seamlessly scale up to the largest databases and workloads.\vspace{1em}
\end{frame}

\section{What is the PostgreSQL Ecosystem?}

\begin{frame}
    \frametitle{What is the PostgreSQL Ecosystem?}

    The PostgreSQL ecosystem describes the software products and services that use or work with PostgreSQL.\pause \\
    \vspace{1em}
    Some examples:
    \vspace{1em}
    \begin{itemize}
        \item pgAdmin - Graphical administration\pause
        \item PostGIS - GIS functions\pause
        \item check\_postgres - Monitoring plugin for Nagios\pause
        \item pg\_partman - Automated partitioning\pause
        \item PgBouncer - Connection pooler\pause
        \item pgBackRest - Backup \& restore\pause
    \end{itemize}
    \vspace{1em}
    And many, many more.
\end{frame}

\begin{frame}
    \frametitle{Why aren't they built into PostgreSQL?}

    There are many possible approaches to a single problem, so the general idea is to have multiple solutions, each tailored to a specific approach. \\
    \vspace{1em}
    This is predicted by Conway's Law, i.e. the community is free to pursue their own ideas, therefore there will be mutliple solutions. \\
    \vspace{1em}
    Over time the best ideas can be integrated into PostgreSQL core, though this process is slow and uncertain.

    % The messaging could stand to be tweaked, but that’s what you get with this governance structure, which for the most part works very well.  - Peter G.
    %
    % Conway's Law -- i.e., the community is free to pursue their own idea, therefore the solution will be multi-part. (https://en.wikipedia.org/wiki/Conway%27s_law)
    %
    % Evidence in support of Conway's law has been published by a team of Massachusetts Institute of Technology (MIT) and Harvard Business School researchers who, using "the mirroring hypothesis" as an equivalent term for Conway's law, found "strong evidence to support [the] mirroring hypothesis", and that "significant differences in [product] modularity" were "consistent with a view that distributed teams tend to develop more modular products".
\end{frame}

\section{Ignoring PostgreSQL}

\begin{frame}
    \frametitle{Ignoring PostgreSQL}

    It is possible to develop software in the PostgreSQL ecosystem without ever interacting with the PostgreSQL community. \\
    \vspace{1em}
    The pgBackRest project did this for several years. \\
    \vspace{1em}
    For many projects this model may work for a long time or indefinitely.
\end{frame}

\section{Contributing to PostgreSQL}

\begin{frame}
    \frametitle{Contributing to PostgreSQL}

    So why did we start contributing to PostgreSQL?
\end{frame}

\begin{frame}
    \frametitle{Reviewing Useful Features}

    \texttt{- Implement backup API functions for non-exclusive backups} \\
    \texttt{Magnus Hagander, PostgreSQL 9.6} \\
    \vspace{1em}
    This patch allowed any external tool to run a non-exclusive backup, which had previously only been possible for the internal \texttt{pg\_basebackup} tool. \\
    \vspace{1em}
    A non-exclusive backup does not write \texttt{backup\_label} into \texttt{\$PGDATA} and does not interfere with other backups in progress.
\end{frame}

\begin{frame}
    \frametitle{Getting Feedback on pgBackRest Features}

    \texttt{- Exclude additional directories in pg\_basebackup} \\
    \texttt{- Exclude pg\_internal.init from BASE\_BACKUP} \\
    \texttt{- Exclude unlogged tables from base backups} \\
    \texttt{- Exclude temp tables from base backups} \\
    \texttt{David Steele, PostgreSQL 10-11} \\
    \vspace{1em}
    These exclusions were all researched and developed for pgBackRest originally.  By contributing them to \texttt{pg\_basebackup} we were able to get a lot of review of the changes.
\end{frame}

\begin{frame}
    \frametitle{Adding a New Feature}

    \texttt{- Allow group access on PGDATA} \\
    \texttt{David Steele, PostgreSQL 11} \\
    \vspace{1em}
    Allows \texttt{\$PGDATA} to have group read permissions so backup can be performed by an unprivileged user.  This was not workable without changes to core. \\
    \vspace{1em}
    This commit depended on: \\
    \vspace{1em}
    \texttt{- Refactor dir/file permissions} \\
    \texttt{David Steele and Adam Brightwell, PostgreSQL 11} \\
    \vspace{1em}
    \texttt{- Refactor new file permission handling} \\
    \texttt{David Steele, PostgreSQL 11} \\
\end{frame}

\section{Choosing a Language}

\begin{frame}
    \frametitle{Choosing a Language}

    pgBackRest was originally written in Perl but we are migrating to C.  Why?\pause \\
    \vspace{1em}
    \begin{itemize}
        \item Speed (mostly startup speed)\pause
        \item Simplified deployment\pause
        \item Ability to use code and data structures from PostgreSQL\pause

        \begin{itemize}
            \item Page Checkums\pause
            \item pg\_control structures\pause
            \item WAL structures\pause
            \item Portability
        \end{itemize}
    \end{itemize}
\end{frame}

\section{Challenges}

\begin{frame}
    \frametitle{Challenge - Removing Exclusive Backup}

    There are a number of reasons why using exclusive backup is not a good idea, but there is a lot of resistance to removing it.  A better alternative (non-exclusive mode) has been available since 9.6.\pause \\
    \vspace{1em}
    Opinions on the patch fell primarily into three (nearly equal) groups:\pause
    \begin{itemize}
        \item No -- it would break too many existing scripts.\pause
        \item Yes -- but not now.\pause
        \item Yes -- we should do this now because a better method has been available for some time.
    \end{itemize}
\end{frame}

\begin{frame}
    \frametitle{Challenge - Improving Backup Documentation}

    In many cases the PostgreSQL backup/archiving documentation is over-simplistic or simply gives bad advice, i.e. using \texttt{cp} for archiving or using exclusive mode backup. \\
    \vspace{1em}
    Updating this documentation with better instructions and discouraging users from using the deprecated exclusive mode backup has been a multi-year process. \\
    \vspace{1em}
    \texttt{- Improve low-level backup documentation.} \\
    \texttt{David Steele, PostgreSQL 11} \\
    \vspace{1em}
    \texttt{- Warn more strongly about the dangers of exclusive backup mode.} \\
    \texttt{David Steele and Robert Haas, PostgreSQL 12}
\end{frame}

\section{Conclusion}

\begin{frame}
    \frametitle{Conclusion}

    Is it worth it? \pause \\
    \vspace{1em}
    Yes.  Working with the community can be challenging but there are many benefits to both projects.
\end{frame}

\section{Questions?}

\begin{frame}
    \frametitle{Questions?}

    website: \url{http://www.pgbackrest.org}\\
    \vspace{1em}
    email: \href{mailto:david@pgbackrest.org}{david@pgbackrest.org} \\
    email: \href{mailto:david@crunchydata.com}{david@crunchydata.com}\\
    \vspace{1em}
    releases: \url{https://github.com/pgbackrest/pgbackrest/releases}\\
    \vspace{1em}
    slides: \url{https://github.com/dwsteele/conference/releases}\\
\end{frame}

% End document
\end{document}
